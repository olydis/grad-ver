For this section framing desicions \tit{do} consider aliasing.

\subsection{New Permissions}

A particular innacuracy of the framing without aliasing approach was the handling of separate access permissions to the heap. In order to incorporate aliasing alongside heap access, we introduce two new permissions:
\begin{itemize}
\item
$\aliased(\set{ x_\alpha })$ is the permission to assume that each $x_\alpha$ is an alias of each other $x_\alpha$.
\item
$\accessed(e.f)$ is the permission to assume that $\cacc(e.f)$ has, separately, been asserted.
\end{itemize}
In the next section the rules for requiring and granting these permissions are detailed. The idea is that $\cacc(e.f)$ formulas will require that the permission context entails that it is possible that $\pnot \accessed(e.f)$, i.e. it is possible that $e.f$ hasn't already been accessed separately. Then $\cacc(e.f)$ grants $\accessed(e.f)$ along with $\accessed(e\p .f)$ for any aliases of what's in $e$'s place.

In addition to keeping track of aliased accesses, the idea for involving $\aliased(\set{ x_\alpha })$ in the permissions is that assertions of aliasing and non-aliasing in expressions (i.e. $x = y \land x \neq y$) can be considered contradictions statically.

% TODO: old
% The rules for framing without aliasing were particularly innacurate because they did not completely handle the separate access permissions to the heap.
% In order to incorporate this, we introduce the permission to separate a part of the heap, written $\available(e.f)$.
% Now the formula $\cacc(e.f)$ requires the $\available(e.f)$ permission and, additionally, grants the $- \ \available(e.f)$ permission which cancles out the ``used up'' $\available(e.f)$ permission.
% As will be shown explicitly in the following algorithms, a formula is checked for framing at the top level with a starting set of permissions including $\available(e.f)$ for every $e.f$, accounting for the fact that the entirety of the heap is available at the top level.

\subsection{Deciding Framing with Aliasing}

% TODO: could allow things that are always false (for example, have no requirements for x = y, even if ~ x = y is in Pi)

Given $\Pi$ a permission set and $\phi$ a formula, the proposition that $\Pi$ \tbf{frames} $\phi$ is written
$$
\Pi \frames \phi
$$
The following algorithm decides $\Pi \frames \phi$.
\begin{align*}
\begin{array}{rclrl}
\Pi \frames \phi & \iff & \tsf{match} \ \phi \ \tsf{with} && \\
%
% expressions
%
&& v &\mt&
  \top
\\
&& x &\mt&
  \top
\\
&& x = y &\mt&
  \Pi \entails \pnot (\pnot \aliased\set{x,y})
\\
&& x \odot y &\mt&
  \Pi \entails \pnot (\pnot (\pnot \aliased\set{x,y}))
\\
&& e_1 \oplus e_2 &\mt&
  (\Pi \merge \grantedPi(e_1) \frames e_1) \land \
  (\Pi \merge \grantedPi(e_2) \frames e_2)
\\
&& e_1 \&\& e_2 &\mt&
  (\Pi \merge \grantedPi(e_1) \frames e_1) \ \land \
  (\Pi \merge \grantedPi(e_2) \frames e_2)
\\
&& e_2 \mid\mid e_2 &\mt&
(\Pi \merge \grantedPi(e_1) \frames e_1) \ \lor \
(\Pi \merge \grantedPi(e_2) \frames e_2)
\\
&& e_2 \odot e_2 &\mt&
(\Pi \merge \grantedPi(e_1) \frames e_1) \ \land \
(\Pi \merge \grantedPi(e_2) \frames e_2)
\\
&& e.f &\mt&
  (\Pi \frames e) \ \land \
  (\Pi \entails \accessed(e.f))
\\
&& \cacc(e.f) &\mt&
  (\Pi \merge \granted(\cacc(e.f)) \frames e.f) \ \land \
  (\Pi \entails \pnot (\pnot (\pnot \accessed(e.f))))
\\
% TODO: should this be split and handled differently in the \ast and \land cases?
&& \phi_1 \cast \phi_2 &\mt&
  \Pi \merge \grantedPi(\phi_1 \cast \phi_2) \frames
  \phi_1, \phi_2
\\
&& \alpha_C(e_1, \dots, e_k) &\mt&
  \Pi \ \merge \ \frames e_1, \dots, e_k
\\
&& \cif \ e \ \cthen \ \phi_1 \ \celse \ \phi_2 &\mt&
  (\Pi \frames e) \ \land \
  (\Pi \merge \grantedPi(e) \frames \phi_1) \\ &&&& \land \
  (\Pi \merge \notgrantedPi(e) \frames \phi_2)
\\
&& \cunfolding \ \alpha_C(\vec{e}) \ \cin \ \phi\p &\mt&
  (\Pi \frames \alpha_C(\vec{e})) \ \land \
  (\Pi \merge \granted(\alpha_C(\vec{e})) \frames \phi\p)
\end{array}
\end{align*}

\begin{align*}
\begin{array}{rclrl}
\tsf{granted}_\Pi(\phi) & := & \tsf{match} \ \phi \ \tsf{with} && \\
%
% expressions
%
&& v &\mt&
  \empty
\\
&& x &\mt&
  \empty
\\
&& x = y &\mt&
  \set{ \aliased(\aliasesPi(x) \cup \aliasesPi(y)) }
  \\ &&&& \ \merge \
  \set{ \pnot \aliased\set{ x\p, \tilde y }
    \mid x\p \in \aliasesPi(x), \tilde y \in \nonaliasesPi(y) }
  \\ &&&& \ \merge \
  \set{ \pnot \aliased\set{ \tilde x, y\p }
    \mid \tilde x \in \nonaliasesPi(x), y\p \in \aliasesPi(y) }
\\
&& x \oplus y &\mt&
  \set{ \pnot \aliased(\set{ x\p } \cup \aliasesPi(y)) \mid x\p \in \aliasesPi(x) }
  \\ &&&& \ \merge \
  \set{ \pnot \aliased(\set{ y\p } \cup \aliasesPi(x)) \mid y\p \in \aliasesPi(x) }
\\
&& e_1 \oplus e_2 &\mt&
  \\
\\
&& e_2 \odot e_2 &\mt&
  %TODO
\\
&& \cacc(x.f) &\mt&
  \set{ \accessed(x\p.f) \mid x\p \in \aliasesPi(x) }
\\
&& \phi_1 \ast \phi_2 &\mt&
  % TODO: define \merge^\ast
  \tsf{granted}_\Pi(\phi_1) \ \merge^\ast \ \tsf{granted}_\Pi(\phi_2)
\\
&& \phi_1 \land \phi_2 &\mt&
  % TODO: define \merge^\land
  \tsf{granted}_\Pi(\phi_1) \ \merge^\land \ \tsf{granted}_\Pi(\phi_2)
\\
&& \alpha_C(\vec{e}) &\mt&
  \set{ \assumed(\alpha_C(\vec{e})) }
\\
&& \cif \ e \ \cthen \ \phi_1 \ \celse \ \phi_2 &\mt&
  (\tsf{granted}_\Pi(e) \merge \tsf{granted}_\Pi(\phi_1)) \ \overlap \
  (\notgrantedPi(e) \ \merge \ \tsf{granted}_\Pi(\phi_2))
\\
&& \cunfolding \ \alpha_C(\vec{e}) \ \cin \ \phi\p &\mt&
  \grantedPi(\alpha_C(\vec{e})) \ \merge \
  \grantedPi(\phi\p)
\end{array}
\end{align*}

\begin{align*}
\notgrantedPi(\phi) := \set{ \pnot \pi \mid \pi \in \grantedPi(\phi) }
\end{align*}

%-----------------------------------------------------------------------------------------------------------------------------
\subsection{Notes}

\begin{itemize}
  \item
  Let $P$ be a proposition.
  Then, $\pnot (\pnot P)$ is informally read as ``it is possible that $P$''.
  Likewise, $\pnot (\pnot (\pnot P))$ is informally read as ``it is possible that $\pnot P$''.

  \item $\merge^\land$ allows for overlapping $\accessed(e.f)$ permissions. The branches are not necessarily working on the heap separately, so its ok.

  \item $\merge^\ast$ does not allow for overrlapping $\accessed(e.f)$ permissions. The branches must work on the heap separately, so such permissions conflict.
\end{itemize}

%-----------------------------------------------------------------------------------------------------------------------------
\subsection{Deciding Self-Framing with Aliasing}

\noindent
The following algorithm decides $\selfframes \phi$ for a given formula $\phi$.
\begin{align*}
\selfframes \phi \iff \empty \frames \phi
\end{align*}
