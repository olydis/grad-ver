\documentclass{article}

\usepackage[letterpaper, margin=1.5in]{geometry}
\usepackage{amsthm, amsmath, amssymb}
\usepackage{titling}

\setlength{\parindent}{1em}
\setlength{\parskip}{1em}

\setlength{\droptitle}{-2em}
\title{Framing Rules}
\author{Henry Blanchette}
\date{}

\newcommand{\tsf}{\textsf}
\newcommand{\tit}{\textit}
\newcommand{\tbf}{\textbf}
\newcommand{\ttt}{\texttt}
\newcommand{\access}{\tsf{access}}
\newcommand{\assume}{\tsf{assume}}
\newcommand{\footprint}[1]{\lfloor #1 \rfloor}
\newcommand{\frames}{\vDash_I}
\newcommand{\selfframes}{\vdash_{\tsf{frm}I}}
\newcommand{\selfframeswith}[1]{\vdash^#1_{\tsf{frm}I}}
\newcommand{\mt}{\mapsto}
\newcommand{\set}[1]{\left\{ #1 \right\}}
\newcommand{\cast}{\circledast}
\renewcommand{\vec}{\overline}
\newcommand{\p}{^\prime}
\renewcommand{\empty}{\varnothing}

% code keywords
\newcommand{\cif}{\ttt{if}}
\newcommand{\cthen}{\ttt{then}}
\newcommand{\celse}{\ttt{else}}
\newcommand{\cacc}{\ttt{acc}}
\newcommand{\cunfolding}{\ttt{unfolding}}
\newcommand{\cin}{\ttt{in}}

% formal keywords
\newcommand{\id}{\tit{id}}
\newcommand{\granted}{\tsf{granted}}
\newcommand{\aliases}{\tsf{aliases}}
\newcommand{\nonaliases}{\tsf{non-aliases}}
\newcommand{\unaliases}{\tsf{undermined-aliases}}
\newcommand{\compatible}{\uplus}

%-----------------------------------------------------------------------------------------------------------------------------
%-----------------------------------------------------------------------------------------------------------------------------
\begin{document}
%-----------------------------------------------------------------------------------------------------------------------------

\maketitle

%-----------------------------------------------------------------------------------------------------------------------------
%-----------------------------------------------------------------------------------------------------------------------------
\section{Definitions}

\textit{Note:} in this document ``formula'' refers to ``precise formula,'' however gradual formulas will eventually be supported.

\noindent
A \textbf{permission} is to either access a field, written $\access(e.f)$, or to assume a predicate holds of its arguments, written $\assume(\alpha_C(\vec{e}))$.

\noindent
A formula $\phi$ \textbf{requires} a permission $\pi$ if $\phi$ contains an access or assumption that $\pi$ premits. The set of all permissions that $\phi$ requires (the set of permissions required to frame $\phi$) is called the \textbf{requirements} of $\phi$.

\noindent
A formula $\phi$ \textbf{grants} permission $\pi$ if it contains an adjuct that yields $\pi$.

\noindent
A set of permissions $\Pi$ \textbf{frames} a formula $\phi$ if and only if $\phi$ requires only permissions contained in $\Pi$, written $$ \Pi \frames \phi. $$

\noindent
The \textbf{footprint} of a formula $\phi$ is the smallest permission mask that frames $\phi$, written $$ \footprint \phi. $$

\noindent
A formula $\phi$ is \textbf{self-framing} if and only if for any set of permissions $\phi$, $\Pi \frames \phi$, written $$ \selfframes \phi. $$
In other words, $\phi$ is self-framing if and only if it grants all the permissions that it requires.

%-----------------------------------------------------------------------------------------------------------------------------
%-----------------------------------------------------------------------------------------------------------------------------
\newpage
\section{Deciding Framing without Aliasing}

For this section framing desicions do not consider aliasing, for the sake of an introduction.

\noindent
The following algorithm decides $\Pi \frames \phi$ for a given set of permissions $\Pi$ and formula $\phi$.
\begin{align*}
\begin{array}{r|lrl}
\Pi \frames \phi
\iff \tsf{match} \ \phi \ \tsf{with}
%
% expressions
%
& v, x                &\mt& \top
\\
& e_1 \oplus e_2      &\mt& \Pi \frames e_1, e_2
\\
& e_1 \odot e_2       &\mt& \Pi \frames e_1, e_2
\\
& e.f                 &\mt& \Pi \frames e
                      \land \cacc(e.f) \in \Pi
\\
% formulas
%
& \cacc(e.f)          &\mt& \Pi \frames e
\\
% ordered version
% & \phi_1 \cast \phi_2 &\mt& \Pi \frames \phi_1
%                       \land \Pi \cup \granted(\phi_1) \frames \phi_2
% unordered version
& \phi_1 \cast \phi_2 &\mt& \Pi \cup \tsf{granted}(\phi_1 \cast \phi_2) \frames \phi_1, \phi_2
\\
& \alpha_C(\vec{e})   &\mt& \Pi \frames \vec{e}
\\
& \cif \ e \ \cthen \ \phi_1 \ \celse \ \phi_2
                      &\mt& \Pi \frames e, \phi_1, \phi_2
\\
& \cunfolding \ \alpha_C(\vec{e}) \ \cin \ \phi
                      &\mt& \assume(\alpha_C(\vec{e})) \in \Pi
                      \land \Pi \frames \alpha_C(\vec{e})
                      \land \Pi \frames \phi
\end{array}
\end{align*}

\noindent
The following algorithm collects the set of permissions granted by a given formula $\phi$.
\begin{align*}
\begin{array}{r|lrl}
\granted(\phi)
:= \tsf{match} \ \phi \ \tsf{with}
%
% expressions
%
& e                   &\mt& \empty \\
%
% formulas
%
& \cacc(e.f)          &\mt& \set{ \access(e.f) }
\\
& \phi_1 \cast \phi_2 &\mt& \granted(\phi_1)
                      \cup  \granted(\phi_2)
\\
& \alpha_C(\vec{e})   &\mt& \set{ \assume(\alpha_C(\vec{e})) }
\\
& \cif \ e \ \cthen \ \phi_1 \ \celse \ \phi_2
                      &\mt& \granted(\phi_1) \cap \granted(\phi_2)
\\
& \cunfolding \ \alpha_C(\vec{e}) \ \cin \ \phi
                      &\mt& \granted(\phi)
\end{array}
\end{align*}

%-----------------------------------------------------------------------------------------------------------------------------
\subsection{Notes}

\begin{itemize}
\item The conditional expression $e$ in a formula of the form $(\cif \ e \ \cthen \ \phi_1 \ \celse \ \phi_2)$ is considered indeteminant for the purposes of statically deciding framing.

\item The body formula $\phi$ in a formula of the form $(\cunfolding \ \cacc_C(\vec{e})) \ \cin \ \phi)$ does not have to make use of the $\assume(\cacc_C(\vec{e}))$ required by the structure.
\end{itemize}

%-----------------------------------------------------------------------------------------------------------------------------
\subsection{Deciding Self-Framing without Aliasing}

\noindent
The following algorithm decides $\selfframes \phi$ for a given formula $\phi$.
\begin{align*}
\selfframes \phi \iff \empty \frames \phi
\end{align*}

%-----------------------------------------------------------------------------------------------------------------------------
%-----------------------------------------------------------------------------------------------------------------------------
\newpage
\section{Aliasing}

The \tbf{alias status} of a set of identifiers $\set{ x_\alpha }$ is exactly one of the following: \aliases, \nonaliases, \tsf{undetermined-aliases}.
\begin{itemize}
\item The $x_\alpha$ are \aliases\ if each $x_\alpha$ refers to the same memory in the heap.
\item The $x_\alpha$ are \nonaliases\ if each $x_\alpha$ refers to distinct memory in the heap.
\item The $x_\alpha$ are \unaliases\ if they may be \aliases\ or \nonaliases.
\end{itemize}

\noindent
An \tbf{alias class} is a pair $[S,I]$ where $S$ is an alias status and $I$ is a set of identifiers where the identifiers of $I$ have alias status $S$.
$\tsf{identifiers}(\set{[S_\alpha, I_\alpha]}) := \bigcup I_\alpha$ is the set of identifiers of a set of alias classes.
It is possible to keep track of \unaliases\ classes. However, for the sake of efficiency, some give identifiers are considered \unaliases\ if no subset of them are asserted as \aliases\ nor \nonaliases\ by any alias class.

\noindent
A set of alias classes $\set{ [S_\alpha, I_\alpha] }$ is \tbf{overlapping} if and only if
$$ \bigcup I_\alpha \neq \empty. $$

\noindent
A set of alias statuses $\set{ S_\alpha }$ is \tbf{compatible} if and only if
$$ \forall S \in \set{S_\alpha} : \forall \alpha : S_\alpha = S $$

\noindent
A set of alias classes $A$ is \tbf{compatible} if and only if
$$
\forall \set{ [S_\alpha, I_\alpha] } \subset A :
\set{ [S_\alpha, I_\alpha] } \ \text{is overlapping}
\implies
\set{ S_\alpha } \ \text{is compatible}
$$
This is to say that a set of compabile alias classes must not assert that a pair of identifiers are both \aliases\ and \nonaliases\ --- every overlapping set of alias classes is compatible.

\noindent
Given two compatible sets of alias classes $A, A\p$, the compatibility of $A \cup A\p$ can be considered, written $A \compatible A\p$. Deciding $A \compatible A\p$ reduces to computing $\tsf{simplify}(A \cup A\p)$ which either preserves compatibility or raises an exception, where
\begin{align*}
\tsf{simplify}(\set{ [S_\alpha, I_\alpha] })
:= &
\set{
  \left[ S, \bigcup I_{\alpha_i} \right]
  \ \mid \
  \forall \alpha :
  \set{ \alpha_i } = \tsf{LOS}(\alpha)
  \land
  ((\forall i : S = S_{\alpha_i}) \lor (\tsf{raise exception}))
},
\\
\tsf{LOS}(\alpha)
:= &
\set{ \alpha_i }, \
\text{the largest subset of} \ \set{ \alpha } \\&
\text{such that} \ \alpha \in \set{ \alpha_i } \
\text{and} \ \set{ I_{\alpha_i} } \ \text{is overlapping}.
\end{align*}
For each set of overlapping alias classes, \tsf{simplify} either combines then or throws an exception.

\newpage
\subsection{Deciding Alias Class Compatibility}
\noindent
A set of alias classes $A$ is \tbf{compatible} with a formula $\phi$ if and only if $A$ is compatible with the aliasing assertions yielded by $\phi$, written $A \compatible \phi$.
The following algorithm decides $A \compatible \phi$.
\begin{align*}
\begin{array}{r|lrl}
A \compatible \phi
\iff \tsf{match} \ \phi \ \tsf{with}
%
% expressions
%
& e                   &\mt& A \compatible \tsf{asserted}(e)
\\
% formulas
%
% TODO: is this right? or is it non-aliases with e.f?
& \cacc(e.f)          &\mt& A \compatible \set{ [\nonaliases, \set{e} \cup \tsf{identifiers}(A) ] }
\\
& \phi_1 \cast \phi_2 &\mt& A \compatible \tsf{asserted}(\phi_1) \compatible \tsf{asserted}(\phi_2)
\\
& \phi_1 \land \phi_2 &\mt& (A \compatible \phi_1) \land (A \compatible \phi_2)
\\
& \cif \ e \
  \cthen \ \phi_1 \
  \celse \ \phi_2     &\mt& \tsf{if}   \ A \compatible \tsf{asserted}(e) \\
                        &&& \ \tsf{then} \ A \compatible         \tsf{asserted}(e) \compatible \tsf{asserted}(\phi_1) \\
                        &&& \ \tsf{else} \ A \compatible \lnot \ \tsf{asserted}(e) \compatible \tsf{asserted}(\phi_2)
\\
& \phi                &\mt& A \compatible \tsf{asserted}(\phi)
\end{array}
\end{align*}
where
\begin{align*}
\lnot A &:= \set{ [\lnot S_\alpha, I_\alpha] \ \mid \ [S_\alpha, I_\alpha] \in A }, \\
\lnot \ \aliases &:= \nonaliases, \\
\lnot \ \nonaliases &:= \aliases.
\end{align*}
The following algorithm collects the set of alias classes asserted by a given formula $\phi$.
\begin{align*}
\begin{array}{r|lrl}
\tsf{asserted}(\phi) := \tsf{match} \ \phi \ \tsf{with}
& x = y               &\mt& \set{ [\aliases, \set{x,y}] }
\\
& x \odot y           &\mt&  \set{ [\nonaliases, \set{e} ] }
\\
& e                   &\mt& \empty
\\
& \cunfolding \
  \alpha(\vec{e}) \
  \cin \ \phi         &\mt& A \compatible \phi
\end{array}
\end{align*}
The following algorithm collects the set of identifiers in a given set of alias classes $A$.
\begin{align*}
\tsf{identifiers}(\set{ [S_\alpha, I_\alpha] }) := \bigcup I_\alpha
\end{align*}

%-----------------------------------------------------------------------------------------------------------------------------
%-----------------------------------------------------------------------------------------------------------------------------
\newpage
\section{Deciding Framing with Aliasing}

For this section framing desicions \tit{do} consider aliasing.
A set of permissions $\Pi$ and a set of alias classes $A$ frames a formula $\phi$ if and only if $\phi$ requires only permission contained in $\Pi$ and $\tsf{alias-classes}(A, \phi)$ is compatible.

\noindent

The following algorithm decides $\Pi \frames \phi$ for a given set of permissions $\Pi$ and formula $\phi$.
\begin{align*}
\begin{array}{r|lrl}
\Pi \frames \phi
\iff \tsf{match} \ \phi \ \tsf{with}
%
% expressions
%
& v, x                &\mt& \top
\\
& e_1 \oplus e_2      &\mt& \Pi \frames e_1, e_2
\\
& e_1 \odot e_2       &\mt& \Pi \frames e_1, e_2
\\
& e.f                 &\mt& \Pi \frames e
                      \land \cacc(e.f) \in \Pi
\\
% formulas
%
& \cacc(e.f)          &\mt& \Pi \frames e
\\
% ordered version
% & \phi_1 \cast \phi_2 &\mt& \Pi \frames \phi_1
%                       \land \Pi \cup \granted(\phi_1) \frames \phi_2
% unordered version
& \phi_1 \cast \phi_2 &\mt& \Pi \cup \tsf{granted}(\phi_1 \cast \phi_2) \frames \phi_1, \phi_2
\\
& \alpha_C(\vec{e})   &\mt& \Pi \frames \vec{e}
\\
& \cif \ e \ \cthen \ \phi_1 \ \celse \ \phi_2
                      &\mt& \Pi \frames e, \phi_1, \phi_2
\\
& \cunfolding \ \alpha_C(\vec{e}) \ \cin \ \phi
                      &\mt& \assume(\alpha_C(\vec{e})) \in \Pi
                      \land \Pi \frames \alpha_C(\vec{e})
                      \land \Pi \frames \phi
\end{array}
\end{align*}

\noindent
The following algorithm collects the set of permissions granted by a given formula $\phi$.
\begin{align*}
\begin{array}{r|lrl}
\granted(\phi)
:= \tsf{match} \ \phi \ \tsf{with}
%
% expressions
%
& e                   &\mt& \empty \\
%
% formulas
%
& \cacc(e.f)          &\mt& \set{ \access(e.f) }
\\
& \phi_1 \cast \phi_2 &\mt& \granted(\phi_1)
                      \cup  \granted(\phi_2)
\\
& \alpha_C(\vec{e})   &\mt& \set{ \assume(\alpha_C(\vec{e})) }
\\
& \cif \ e \ \cthen \ \phi_1 \ \celse \ \phi_2
                      &\mt& \granted(\phi_1) \cap \granted(\phi_2)
\\
& \cunfolding \ \alpha_C(\vec{e}) \ \cin \ \phi
                      &\mt& \granted(\phi)
\end{array}
\end{align*}

%-----------------------------------------------------------------------------------------------------------------------------
\subsection{Notes}

\begin{itemize}
\item The conditional expression $e$ in a formula of the form $(\cif \ e \ \cthen \ \phi_1 \ \celse \ \phi_2)$ is considered indeteminant for the purposes of statically deciding framing.

\item The body formula $\phi$ in a formula of the form $(\cunfolding \ \cacc_C(\vec{e})) \ \cin \ \phi)$ does not have to make use of the $\assume(\cacc_C(\vec{e}))$ required by the structure.
\end{itemize}

%-----------------------------------------------------------------------------------------------------------------------------
\subsection{Deciding Self-Framing with Aliasing}

\noindent
The following algorithm decides $\selfframes \phi$ for a given formula $\phi$.
\begin{align*}
\selfframes \phi \iff \empty \frames \phi
\end{align*}

%-----------------------------------------------------------------------------------------------------------------------------
\end{document}
%-----------------------------------------------------------------------------------------------------------------------------
%-----------------------------------------------------------------------------------------------------------------------------
