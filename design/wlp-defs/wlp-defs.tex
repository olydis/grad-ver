\documentclass {article}

\usepackage[letterpaper]{geometry}
\usepackage{amsthm, amsmath, amssymb, stmaryrd}
\usepackage{mathpartir}
 
\newtheorem{theorem}{Theorem}[section]
\newtheorem{lemma}[theorem]{Lemma}

\title {Verifier WLP Definitions}
\author {Jenna Wise, Johannes Bader, Jonathan Aldrich, \'{E}ric Tanter}
\date {\today}

%% Commands
\newcommand{\lcar}{\left<}
\newcommand{\rcar}{\right>}
\newcommand{\true}{\text{true}}
\newcommand{\eif}[3]{if \ ( #1 ) \ \{ #2 \} \ else \ \{#3\}}
\newcommand{\fphi}{\widehat{\phi}}
\newcommand{\tphi}{\widetilde{\phi}}
\newcommand{\acc}[1]{\text{acc}(#1)}
\newcommand{\imp}{\Rightarrow}
\newcommand{\timp}{\ \widetilde{\Rightarrow}\ }
\newcommand{\maximp}[2]{\underset{\Rightarrow}{\text{max}}\left\{#1 \mid #2\right\}}

\newcommand{\wlp}[2]{\text{WLP}(#1,#2)}
\newcommand{\twlp}[2]{\widetilde{\text{WLP}}(#1,#2)}
\newcommand{\swlp}[2]{\text{sWLP}(#1,#2)}
\newcommand{\swlpi}[2]{\text{sWLP}_i(#1,#2)}

% uppercase word defs
\newcommand{\satdef}{\textsc{SatFormula}}
\newcommand{\implsdef}{\textsc{ImplStatic}}
\newcommand{\implgdef}{\textsc{ImplGrad}}

\begin{document}

\maketitle

%\begin{figure*}[ht!]
%\begin{flalign*}
%%\spc{skip}{\phi} = \phi
%%\and
%%\spc{s_1;s_2}{\phi} = \spc{s_2}{\spc{s_1}{\phi}}
%%\and
%%\spc{T\ x}{\exists \overline{x}.\theta \wedge \psi} = \exists \overline{x}.(\theta \wedge x = \text{defaultValue}(T)) \wedge \psi
%%\and
%%\spc{x := e}{\exists \overline{x}.\theta \wedge \psi} = \exists \overline{x},x' . ([x'/x]\theta \wedge x = [x'/x]e) \wedge [x'/x]\psi
%%\and
%%\spc{x := y.f}{\exists \overline{x}.\theta \wedge (\psi \ast y.f \mapsto e)} =  \exists \overline{x},x' . ([x'/x]\theta \wedge x = [x'/x]e) \wedge [x'/x](\psi \ast y.f \mapsto e)
%%\and
%%\spc{x.f := y}{\exists \overline{x}.\theta \wedge (\psi \ast x.f \mapsto e)} = \exists \overline{x}.\theta \wedge (\psi \ast x.f \mapsto y)\\
%%\and
%%\spc{x := new\ C}{\exists \overline{x}.\theta \wedge \psi} = \exists \overline{x},x' . [x'/x]\theta \wedge ([x'/x]\psi\ \overline{\ast\ x.f_i \mapsto \text{defaultValue}(T_i)})\ \text{where fields}(C)=\overline{T_i\ f_i} \\
%%\and
%%\spc{y := z.m(\overline{x})}{\exists \overline{y}.\theta \wedge \psi} = \begin{cases} \exists \overline{y}, y' . ([y'/y]\theta_{frame} \wedge \theta_Q) \wedge ([y'/y]\psi_{frame} \ast \psi_Q) & \text{if}\ \exists \overline{y}.(\theta \wedge \psi) \imp \exists \overline{y}.(\theta_{frame} \wedge \theta_P) \wedge (\psi_{frame} \ast \psi_P) \\ undefined & \text{otherwise} \end{cases}
%%\\ \text{where}\ \left[z/this,\overline{x_i/\text{params}(m)_i}\right]\text{pre(m)}=\theta_P \wedge \psi_P\ \text{and}\ \left[z/this, y/result, \overline{x_i/\text{old}(\text{params}(m)_i)}\right]\text{post(m)}=\theta_Q \wedge \psi_Q 
%%\and
%%\spc{assert\ \phi_a}{\phi} = \begin{cases} \phi & \text{if}\ \phi \imp \phi_a \\ undefined & \text{otherwise} \end{cases}
%%% WLP
%\wlp{skip}{\phi} &= \phi & \\
%\wlp{s_1;s_2}{\phi} &= \wlp{s_1}{\wlp{s_2}{\phi}}& \\
%\wlp{T\ x}{\phi} &= \phi &\\
%\wlp{x := e}{\phi} &= [e/x]\phi &\\
%\wlp{x := y.f}{\theta \wedge \psi} &= [e/x]\theta \wedge ([e/x]\psi \ast y.f \rightarrow e) &\\
%\wlp{x.f := y}{\theta \wedge \psi} &= \maximp{\theta' \wedge \psi'}{\theta' \wedge (\psi' \ast x.f \rightarrow y) \imp \theta \wedge \psi} &\\
%\wlp{x := new\ C}{\phi} &= &\\
%\wlp{y := z.m(\overline{x})}{\phi} &= &\\
%\wlp{assert\ \phi}{\phi} &=  &\\
%\end{flalign*}
%%\caption{Weakest Liberal Precondition Calculus}
%%\label{wp-rules}
%\end{figure*}

\section{Weakest liberal precondition calculus definitions over self-framed non-gradual formulas}
\hspace{0.5cm}

$\wlp{skip}{\fphi} = \fphi  $

\vspace{0.5cm}

$\wlp{s_1;s_2}{\fphi} = \wlp{s_1}{\wlp{s_2}{\fphi}} $

\vspace{0.5cm}

%$\wlp{T\ x}{\fphi} = \fphi\left[\text{defaultValue}(T)/x\right] $ -- NEEDS TO CHANGE

%\vspace{0.5cm}

$\wlp{T \ x := e}{\fphi} =\maximp{\fphi'}{\fphi' \imp \fphi[e/x] \quad \wedge \quad \fphi' \imp \acc{e}} $

\vspace{0.5cm}

$\wlp{\eif{x \odot y}{s_1}{s_2}}{\fphi} = $
%\underset{\Rightarrow}{\text{max}} \Big{\{} \fphi' \mid \fphi' \imp \text{acc}(x) \quad \wedge \quad \fphi' \imp \text{acc}(y) \quad \wedge \quad$ 

%\indent  \hspace{4cm} $\fphi' \imp \left(\left(x \odot y \imp \wlp{s_1}{\fphi}\right) \quad \wedge \quad \left(\neg(x \odot y) \imp \wlp{s_2}{\fphi}\right)\right) \Big{\}}$

\vspace{0.5cm}

$\wlp{x.f := y}{\fphi} = \acc{x.f} \ \ast \ \maximp{\fphi'}{\fphi' \ast \acc{x.f} \ast (x.f = y) \imp \fphi \ \wedge \ \fphi' * \acc{x.f} \in \satdef} $

\vspace{0.5cm}

$\wlp{x := new\ C}{\fphi} = \maximp{\fphi'}{ \fphi' \ast (x \neq null) \ast \overline{\acc{x.f_i}} \imp \fphi}$

%\overline{\acc{x.f_i} \ast (x.f_i = \text{defaultValue}(T_i))} 

\indent  \hspace{4cm} where fields$(C) = \overline{T_i \ f_i}$

\vspace{0.5cm}

$\wlp{y := z.m(\overline{x})}{\fphi} = undefined$

\vspace{0.5cm}

$\wlp{y := z.m_C(\overline{x})}{\fphi} = \underset{\Rightarrow}{\text{max}} \Big{\{} \fphi' \mid y \not \in \text{FV}(\fphi') \quad \wedge \quad \fphi' \imp (z \neq null) \ \ast \ \text{pre}(C,m)\left[z/this, \overline{x_i/\text{params}(C,m)_i}\right] $ 
\indent \hspace{4cm} $ \wedge \quad \fphi' \ast \text{post}(C,m)\left[z/this, \overline{x_i/\text{old}(\text{params}(C,m)_i)}, y/result \right] \imp \fphi \Big{\}}$

\vspace{0.5cm}

$\wlp{assert\ \phi_a}{\fphi} = \maximp{\fphi'}{\fphi' \imp \fphi \quad \wedge \quad \fphi' \imp \phi_a} $

\vspace{0.5cm}

$\wlp{release \ \phi_a}{\fphi} = $

\vspace{0.5cm}

$\wlp{hold \ \phi_a \ \{s\}}{\fphi} = $

\vspace{0.5cm}

\noindent \textbf{Note:} 

\textbf{Dynamic method calls.} Dynamic method calls are left undefined, because we are not verifying programs with dynamic dispatch at this time (all method calls should be static method calls). They are included in the grammar for future implementation.

\textbf{If \& Release \& hold.} Definitions coming soon.

\textbf{Predicates in the logic.} Although the grammar allows for abstract predicate families, we do not support them yet. Therefore, we assume formulas look like:

$$ \phi ::= \true \mid e \odot e \mid acc(e.f) \mid \phi \ast \phi $$

\section{Algorithmic WLP calculus definitions over self-framed non-gradual formulas}
\hspace{0.5cm}

\textbf{Note:} 

It may be helpful to check that the $\wlp{s}{\fphi}$ is well-formed and/or self-framed for some of the more complicated rules, which may be buggy in implementation.

\vspace{0.5cm}

\textbf{Note:}

We do not always calculate the correct WLP due to a limitation in the expressiveness of our specification language. Concrete formulas which do not contain enough information to determine whether two or more objects alias, wrongly assume that those two or more objects do not alias. This is because our current specification language is missing a logical OR or a conditional construct, which would allow us to write self-framed concrete formulas that capture the unknown aliasing between two or more objects. A conditional construct will be added and the definitions below adjusted in the near future.
% the weakest liberal precondition for assert, because concrete formulas cannot support the true weakest liberal precondition. Concrete formulas need to support logical OR. This issue occurs when aliasing constructs are missing from certain concrete formulas; in these cases, there is no way to tell whether accessibility predicates should be conjoined or one of them removed due to aliasing.

\vspace{0.5cm}

$\wlp{skip}{\fphi} = \fphi  $

\vspace{0.5cm}

$\wlp{s_1;s_2}{\fphi} = \wlp{s_1}{\wlp{s_2}{\fphi}} $

\vspace{0.5cm}

$\wlp{T \ x := e}{\fphi} = 
	\begin{cases}
	 \fphi[e/x] & if \ \fphi[e/x] \imp acc(e) \\
	 acc(e) \ast \fphi[e/x] & otherwise
	\end{cases} $

Check that $\wlp{T \ x := e}{\fphi} \ast x = e \imp \fphi$ and that $\wlp{T \ x := e}{\fphi}$ is satisfiable.

\vspace{0.5cm}

$\wlp{\eif{x \odot y}{s_1}{s_2}}{\fphi} = $

\vspace{0.5cm}

$\wlp{x.f := y}{\fphi} =  
	\begin{cases}
	 \fphi[y/x.f] & if \ \fphi[y/x.f] \imp acc(x.f) \\
	 acc(x.f) \ast \fphi[y/x.f] & otherwise
	\end{cases}$

Check that $\wlp{x.f := y}{\fphi} \ast x.f = y \imp \fphi$ and that $\wlp{x.f := y}{\fphi}$ is satisfiable.

\textbf{Important cases to consider:}

$\fphi = acc(x.f) \ast x.f = p \ast x.f = q \ast a = b $

$\fphi = acc(x.f) \ast acc(x.f.f) \ast x = y $

\vspace{0.5cm}

$\wlp{x := new\ C}{\fphi} = 
	\begin{cases}
	 \fphi \div x & if \ (\fphi \div x) \ast x \neq null \ast \overline{x \neq e_i} \ast \overline{acc(x.f_i)} \imp \fphi \\
	 undefined & otherwise
	\end{cases}$ 

where fields$(C) = \overline{T_i \ f_i}$, $\fphi \div x$ means to transitively expand (in-)equalities ($\odot$) and then remove conjunctive terms containing $x$, and $\overline{x \neq e_i}$ are conjunctive terms in $\fphi$. 

Check $\wlp{x := new\ C}{\fphi}$ is satisfiable.

\textbf{Important cases to consider:}

$\fphi = x \neq null \ast acc(x.f)$

$\fphi = x \neq null \ast acc(x.f) \ast x.f = 1 \ast x.f = y$ --- should fail, bad postcondition

$\fphi = x \neq null \ast acc(x.f) \ast x = y \ast x = z$ --- should fail, bad postcondition

$\fphi = x \neq null \ast acc(x.f) \ast x = y \ast y = z$ --- should fail, bad postcondition

$\fphi = x \neq null \ast acc(x.f) \ast x \neq y \ast y = z$

$\fphi = x \neq null \ast acc(x.f) \ast acc(x.f.f) \ast x.f.f \neq y$ --- should fail, bad postcondition

$\fphi = x \neq null \ast acc(y.f) \ast x = y$ --- should fail, bad postcondition

$\fphi = x \neq null \ast acc(x.f) \ast y > x.f \ast x.f > z \ast r \geq x.f \ast x.f \geq s$ --- should fail, bad postcondition


\textbf{Note:}

$x := new \ C$ creates a fresh object and assigns it to $x$ without setting default values to the object's fields; therefore, postconditions cannot say anything about the value of $x$ other than it does not equal other values (no aliasing with $x$) and they cannot say anything about the values of the fields of $x$.

\vspace{0.5cm}

$\wlp{y := z.m(\overline{x})}{\fphi} = undefined$

\vspace{0.5cm}

$\wlp{y := z.m_C(\overline{x})}{\fphi} = \fphi \ \div \ y \overline{\ \div \ (e.f)_i} \ast z \neq null \ast \text{pre}(C,m)\left[z/this, \overline{x_i/\text{params}(C,m)_i}\right]$

%\fphi_{acc} \ \ast \mid \fphi \ \div \ y \overline{\ \div \ (x.f)_i} \mid \ \ast \ z \neq null \ \ast \ \left\vert \text{pre}(C,m)\left[z/this, \overline{x_i/\text{params}(C,m)_i}\right] \right\vert$

where $\phi \div x$ is defined differently than for the allocation rule above. $\phi \div x$ means to transitively expand (in-)equalities and comparisons and then remove conjunctive terms containing $x$, but removal of conjuctive terms containing $x$ as an object dereference should replace $x$ with an alias, if an alias exists. If an alias does not exist, then the term is removed as usual.

$(e.f)_i$ and $\phi \overline{\ \div \ (e.f)_i}$ are defined as follows:

$(e.f)_i$ is a field access used in $m$. It comes from a list of field accesses used in $m$ that is determined in the following way:
\begin{enumerate}
%\item Remove conjunctive terms containing field accesses in $\vert \fphi \div y \vert$, call this resulting formula $\vert \fphi \div y \vert'$
\item Determine the list of field accesses occurring in $\text{pre}(C,m)\left[z/this, \overline{x_i/\text{params}(C,m)_i}\right]$
	\begin{itemize}
	\item This list should be determined by starting with the ones in the accessibility predicates and then adding their duplicates due to aliasing to the list. If the static part of 
	
	$\text{pre}(C,m)\left[z/this, \overline{x_i/\text{params}(C,m)_i}\right]$ is not self-framed, as will be the case in the WLP definition of method calls over gradual formulas, then start with the ones in the accessibility predicates and add the ones which occur in the non-accessibility part of 
	
	$\text{pre}(C,m)\left[z/this, \overline{x_i/\text{params}(C,m)_i}\right]$ that are not already in the list. Finally, their duplicates due to aliasing should be added to the list.
	\end{itemize}
\end{enumerate} 
%determined from the accessibility predicates in the substituted precondition of $m$ (including duplicates due to aliasing),

$\phi \overline{\ \div \ (e.f)_i}$ removes dependences on the field accesses in the removal list computed above in the following way:
\begin{enumerate}
\item Transitively expand (in-)equalities and comparisons in $\phi$, so $\phi$ is now modified in this way
\item Determine the list of aliases to each variable or field access dereferenced to a final field value in $\phi$ using $\phi$
\item Use the list of aliases to each variable or field access dereferenced to a final field value in $\phi$ to replace any variable or field access in $\phi$ that aliases to a top-level variable being dereferenced in a field access (ie. $x$ in $x.f$ and $x.f.g$) in the list of removal that the $(e.f)_i$s come from. When substituting in multi-level field accesses (ie. $x.f.g.h$), the top-level object dereferences take precedence (ie. $x$ should be substituted if possible; if not possible for $x$, then $x.f$ should be next and so forth).
\item Remove conjunctive terms containing the $(e.f)_i$s from $\phi$
	\begin{itemize}
	\item Removal for all conjunctive terms is straightforward (remove the term containing it) except for conjunctive terms which contain the $(e.f)_i$s as an object being dereferenced. In this case, each $(e.f)_i$ should be substituted with an available alias in $\phi$; if no alias exists, then remove the term completely. The available alias should not be used if it exists in the list for removal that the $(e.f)_i$s come from or if the substitution of the alias results in a field access which contains or may be itself a field access that exists in the list for removal that the $(e.f)_i$s come from.
	\item The order in which the $(e.f)_i$s are removed from $\phi$ should be from the smallest to largest in length of dereferencing, ie. $x.f$ is removed before $x.f.h$ which is removed before $x.f.h.g$ and so forth; $(e.f)_i$s with lengths of two are removed before $(e.f)_i$s with lengths of three and so forth.
	\end{itemize}
\end{enumerate}


%and $\fphi_{acc}$ is a special formula of accessibility predicates that is described below (this description is similar to how computation for $\fphi_{acc}$ is done for the assert rule following this rule, so some details are left out here):

%\begin{enumerate}
%\item Start with a list of the accessibility predicates in $\fphi \div y \overline{\ \div \ (x.f)_i}$ and 
%
%$\text{pre}(C,m)\left[z/this, \overline{x_i/\text{params}(C,m)_i}\right]$
%%\item Remove any accessibility predicate containing $y$
%%\item Add to the list the accessibility predicates in $\text{pre}(C,m)\left[z/this, \overline{x_i/\text{params}(C,m)_i}\right]$
%\item Add aliasing information from the actual conjoined accessibility predicates in $\fphi \div y \overline{\ \div \ (x.f)_i}$ and 
%
%$\text{pre}(C,m)\left[z/this, \overline{x_i/\text{params}(C,m)_i}\right]$ to the non-accessibility parts of $\fphi \div y \overline{\ \div \ (x.f)_i}$ and 
%
%$\text{pre}(C,m)\left[z/this, \overline{x_i/\text{params}(C,m)_i}\right]$ respectively. In other words, the accessibility predicates in $\fphi \div y \overline{\ \div \ (x.f)_i}$ $\left(\text{pre}(C,m)\left[z/this, \overline{x_i/\text{params}(C,m)_i}\right]\right)$ separated by the separating conjunction indicate that if the fields ultimately being accessed within two or more of them are the same, then the objects being dereferenced in those field accesses do not alias, so add this information to the non-accessibility part of $\fphi \div y \overline{\ \div \ (x.f)_i}$ $\left(\text{pre}(C,m)\left[z/this, \overline{x_i/\text{params}(C,m)_i}\right]\right)$ to not lose information
%\item Determine the list of aliases and the list of non-aliases to each variable or field access dereferenced to a final field value in the current list of accessibility predicates using 
%
%$\mid \fphi \div y \overline{\ \div \ (x.f)_i} \mid \ast \left\vert \text{pre}(C,m)\left[z/this, \overline{x_i/\text{params}(C,m)_i}\right] \right\vert$
%\item Remove duplicate accessibility predicates, including accessibility predicates which are duplicate due to aliasing (the list of aliases and list of non-aliases for each variable and field access dereferenced to a final field value in an acc predicate should help)
%	\begin{enumerate}
%	\item If it is unknown whether two objects alias due to missing or conflicting information in $\fphi \div y \overline{\ \div \ (x.f)_i}$ and $\text{pre}(C,m)\left[z/this, \overline{x_i/\text{params}(C,m)_i}\right]$, then assume (for now) that they do not alias
%	\end{enumerate}
%\item $\fphi_{acc}$ is the list of non-duplicated accessibility predicates conjoined with the separating conjunction
%\end{enumerate}

Also, check that 

$\wlp{y := z.m_C(\overline{x})}{\fphi} \overline{\ \div \ (e.f)_i} \ast \text{post}(C,m)\left[z/this, \overline{x_i/\text{old}(\text{params}(C,m)_i)}, y/result \right] \imp \fphi$,

and
%$\fphi'_{acc} \ \ast \ \left\vert \wlp{y := z.m_C(\overline{x})}{\fphi} \overline{\div (x.f)_i} \right\vert \ \ast \ \left\vert \text{post}(C,m)\left[z/this, \overline{x_i/\text{old}(\text{params}(C,m)_i)}, y/result \right] \right\vert$ 

%\hspace{1cm} $\imp \fphi$, and

$\wlp{y := z.m_C(\overline{x})}{\fphi}$ is satisfiable. 

%(\textbf{Note:} Can add extra check of $\wlp{y := z.m_C(\overline{x})}{\fphi} \imp z \neq null \ \ast \ \text{pre}(C,m)\left[z/this, \overline{x_i/\text{params}(C,m)_i}\right]$ to be safe).

where $(e.f)_i$ is defined as above (coming from the same list as above) and $\phi \overline{\ \div \ (e.f)_i}$ is defined as below:

\begin{enumerate}
\item Transitively expand (in-)equalities and comparisons in $\phi$, so $\phi$ is now modified in this way
\item Remove conjunctive terms containing the $(e.f)_i$s as a whole. Do not remove conjunctive terms that contain an $(e.f)_i$ as an object being dereferenced
\end{enumerate}

%and $\fphi'_{acc}$ is determined in a similar fashion as describe for $\fphi_{acc}$ by replacing $\fphi \div y \overline{\ \div \ (x.f)_i}$ with $\wlp{y := z.m_C(\overline{x})}{\fphi} \overline{\div (x.f)_i}$ and $\text{pre}(C,m)\left[z/this, \overline{x_i/\text{params}(C,m)_i}\right]$ with $\text{post}(C,m)\left[z/this, \overline{x_i/\text{old}(\text{params}(C,m)_i)}, y/result \right]$.

\textbf{Important cases to consider:}

TBD -- have them just need to type them up
%What if $y = x_i$ for some argument $x_i$, $z$ could also be an argument, and $y = z$
%$\fphi = $
%
%POST $= acc(result.f) \ast acc(result.f.g) \ast x = result.f \ast z = result$

\textbf{Note:}

This rule assumes only $old(id)$ can show up in the postcondition of a method. $old(.)$ of anything else is not allowed, ie. $old(e.f)$ (for now).

\vspace{0.5cm}

$\wlp{assert\ \phi_a}{\fphi} = \fphi_{acc} \ \ast \mid \phi_a \mid \ast \mid \fphi \mid$


where $\mid \phi \mid$ means the formula $\phi$ without accessibility predicates and $\fphi_{acc}$ is the self-framed formula which contains the accessibility predicates that frame $\mid \phi_a \mid \ast \mid \fphi \mid$ and, in general, the accessibility predicates in $\phi_a$ and $\fphi$ that are not duplicated (even with respect to aliasing).

Also, check $\wlp{assert\ \phi_a}{\fphi} \imp \phi_a$, $\wlp{assert\ \phi_a}{\fphi} \imp \fphi$, and $\wlp{assert\ \phi_a}{\fphi}$ is satisfiable.

\textbf{Advice on how to compute $\fphi_{acc}$:}
\begin{enumerate}
% \item Transitively expand equalities in $\mid \phi_a \mid \ast \mid \fphi \mid$
\item Start with a list of the accessibility predicates in $\phi_a$ and $\fphi$
\item Add accessibility predicates to this list for self-framing $\phi_a$, as some may be missing since $\phi_a$ is not self-framed
	\begin{enumerate}
	\item Duplicates (including those due to aliasing) are fine, because they will be removed later
	\item But, not including obvious duplicates will improve performance
	\end{enumerate}
\item Add aliasing information from the actual conjoined accessibility predicates in $\phi_a$ and $\fphi$ to the non-accessibility parts of $\phi_a$ and $\fphi$ respectively. In other words, the accessibility predicates in $\phi_a$ ($\fphi$) separated by the separating conjunction indicate that if the fields ultimately being accessed within two or more of them are the same, then the objects being dereferenced in those field accesses do not alias, so add this information to the non-accessibility part of $\phi_a$ ($\fphi$) to not lose information
	\begin{enumerate}
	\item $acc(x.f) \ast acc(y.f) \ast x.f = 2$ implies $acc(x.f) \ast acc(y.f) \ast x.f = 2 \ast x \neq y$
	\item $acc(x.f) \ast acc(x.f.g) \ast acc(y.g) \ast acc(y.g.f) \ast x.f = 2$ implies \\ $acc(x.f) \ast acc(x.f.g) \ast acc(y.g) \ast acc(y.g.f) * x.f = 2 \ast x \neq y.g \ast y \neq x.f$
	\end{enumerate}
\item Determine the list of aliases and the list of non-aliases to each variable or field access dereferenced to a final field value in the current list of accessibility predicates using $\mid \phi_a \mid \ast \mid \fphi \mid$
	\begin{enumerate}
	\item Keep in mind that if $x.f \neq y.f$ then $x \neq y$ and that $x.f \neq y.f$ can be determined in many different ways, including through $x.f = 3 * y.f = 4$, so the SMT solver should be helpful and necessary for this
	\end{enumerate}
% Scan the conjunctive terms in $\mid \phi_a \mid \ast \mid \fphi \mid$ for field accesses
\item Remove duplicate accessibility predicates, including accessibility predicates which are duplicate due to aliasing (the list of aliases and list of non-aliases for each variable and field access dereferenced to a final field value in an acc predicate should help)
	\begin{enumerate}
	\item If it is unknown whether two objects alias due to missing or conflicting information in $\phi_a$ and $\fphi$, then assume (for now) that they do not alias
	\end{enumerate}
% For each field access, check if an accessibility predicate exists for it in $\fphi_{acc}$ (including if an aliased accessibility predicate exists for it; the list of aliases for each variable should help)
	% \begin{itemize}
	% \item If so, continue scanning for field accesses
	% \item If not, add the accessibility predicate for the field access to $\fphi_{acc}$ using the separating conjunction
	% \end{itemize}
\item $\fphi_{acc}$ is the list of non-duplicated accessibility predicates conjoined with the separating conjunction
\end{enumerate}

\textbf{Important cases to consider:}

$\mid \phi_a \mid \ast \mid \fphi \mid \ = x \neq null \ast x.f = 1 \ast x.f = y.f \ast y.f \neq p \ast y.f.f > 1$

$\mid \phi_a \mid \ast \mid \fphi \mid \ = x = y \ast x = z \ast x.f = 8 \ast y.f > 4$

$\mid \phi_a \mid \ast \mid \fphi \mid \ = x = y \ast y = z \ast z.f + 1 \leq 10 \ast \true$

$\mid \phi_a \mid \ast \mid \fphi \mid \ = x.f.f \neq y \ast x \neq p \ast p = q \ast x.f > y \ast y > z$

$\mid \phi_a \mid \ast \mid \fphi \mid \ = y > x.f \ast x.f > z \ast r \geq x.f \ast x.f \geq s$

$\phi_a = acc(x.f) \ast y = 2$ and $\fphi = acc(z.f) \ast z.f = 4$

$\phi_a = acc(x.f) \ast x.f = 4$ and $\fphi = acc(z.f) \ast z.f = 4$

\textbf{Assumptions:}

We assume that objects are only referred to in formulas through variables or as field accesses, variables or field accesses which refer to objects are not used in binary operations ($\oplus$), and variables or field accesses which refer to objects are not used in comparison operators other than $\neq$ and $=$.

%	\begin{cases}
%	 \fphi \ast \phi_a & if \ fp_s(\fphi') \cap fp_s(\phi_a') = \emptyset \\
%	 \overline{acc((x.f)_i)} \ast \fphi - F \ast \phi_a - F & otherwise
%	\end{cases}$
%	
%where $\lcar x,f \rcar_i \in F = fp_s(\fphi') \cap fp_s(\phi_a')$ and $\phi - F$ means to remove the accessibility predicates which apply to elements of $F$.
%
%Also, $\fphi'$ and $\phi_a'$ are the transitively expanded versions of the corresponding formula with additional accessibility predicates added as necessary due to aliasing (see important cases below).
%
%\textbf{Important cases to consider:}
%
%$\fphi = acc(x.f) \ast x = p \ast p = q \ast a = b $
%
%--- becomes $\fphi' = acc(x.f) \ast x = p \ast p = q \ast x = q \ast a = b $
%
%--- and finally $\fphi' = acc(x.f) \ast acc(p.f) \ast acc(q.f) \ast x = p \ast p = q \ast x = q \ast a = b $
%
%$\fphi = acc(x.f) \ast x = y $ --- results in $\fphi' = acc(x.f) \ast acc(y.f) \ast x = y $

\vspace{0.5cm}

$\wlp{release \ \phi_a}{\fphi} = $

\vspace{0.5cm}

$\wlp{hold \ \phi_a \ \{s\}}{\fphi} = $

\section{Algorithmic WLP calculus definitions over gradual formulas}
\hspace{0.5cm}

$\twlp{s}{\fphi} = \wlp{s}{\fphi} $ where $s$ is not a call statement

\vspace{0.5cm}

$\twlp{skip}{? \ast \phi} = \ ? \ast \phi$

\vspace{0.5cm}

$\twlp{s_1;s_2}{? \ast \phi} = \twlp{s_1}{\twlp{s_2}{? \ast \phi}} $

\vspace{0.5cm}

$\twlp{T \ x := e}{? \ast \phi} = \ ? \ast \phi[e/x] \ast e = e$

\textbf{Important cases to consider:}

$? \ast \phi = \ ? \ast p = q \ast y.f = 2$ --- no mention of $e$ or $x$ for substitution; need extra $\ast \ e = e$ because of this case

\textbf{OR}

$\twlp{T \ x := e}{? \ast \phi} =
	 \begin{cases}
	 ? \ast \phi[e/x] & if \ \phi[e/x] \imp acc(e) \\
	 ? \ast acc(e) \ast \phi[e/x] & otherwise
	\end{cases}$
	
Check that $\twlp{T \ x := e}{? \ast \phi} \ast x = e \timp ? \ast \phi$ and that $\twlp{T \ x := e}{? \ast \phi}$ is satisfiable.

\vspace{0.5cm}

%$\twlp{\eif{x \odot y}{s_1}{s_2}}{? \ast \phi} = $
%
%\vspace{0.5cm}

$\twlp{x.f := y}{? \ast \phi} =
	\begin{cases}
	 ? \ast \phi[y/x.f] & if \ \phi[y/x.f] \imp acc(x.f) \\
	 ? \ast acc(x.f) \ast \phi[y/x.f] & otherwise
	\end{cases}$

Check that $\twlp{x.f := y}{? \ast \phi} \ast x.f = y \timp \ ? \ast \phi$ and that $\twlp{x.f := y}{? \ast \phi}$ is satisfiable.

\textbf{Important cases to consider:}

$? \ast \phi = \ ? \ast acc(x.f) \ast acc(x.f.f) \ast x = y $

\vspace{0.5cm}

$\twlp{x := new\ C}{? \ast \phi} = 
	\begin{cases}
	 ? \ast \phi \div x & if \ (\phi \div x) \ast x \neq null \ast \overline{x \neq e_i} \ast \overline{acc(x.f_i)} \imp \phi \\
	 undefined & otherwise
	\end{cases}$ 

where fields$(C) = \overline{T_i \ f_i}$, $\phi \div x$ means to transitively expand (in-)equalities ($\odot$) and then remove conjunctive terms containing $x$, and $\overline{x \neq e_i}$ are conjunctive terms in $\phi$. 

Check that $\twlp{x := new\ C}{? \ast \phi} \ast x \neq null \ast \overline{x \neq e_i} \ast \overline{acc(x.f_i)} \timp ? \ast \phi$ and that $\twlp{x := new\ C}{? \ast \phi}$ is satisfiable.

\textbf{Important cases to consider:}

Similar to the ones for the non-gradual version; replacing $\fphi$ with $\phi$.

\vspace{0.5cm}

$\twlp{y := z.m(\overline{x})}{\tphi} = undefined$

\vspace{0.5cm}

$\twlp{y := z.m_C(\overline{x})}{\tphi} = 
	\begin{cases}
		& if \ \tphi, \ \text{pre, post are all concrete formulas} \\
		& if \ \tphi \ \text{concrete and pre or post gradual} \\
		& if \ \tphi \ \text{is gradual and pre or post gradual}
	\end{cases}
$ --- IN PROGRESS

\vspace{0.5cm}

$\twlp{assert\ \phi_a}{? \ast \phi} = \ ? \ast \wlp{assert\ \phi_a}{\phi}$

Check that $\twlp{assert\ \phi_a}{? \ast \phi} \timp ? \ast \phi$, $\twlp{assert\ \phi_a}{? \ast \phi} \timp \phi_a$, and that 

\noindent $\twlp{assert\ \phi_a}{? \ast \phi}$ is satisfiable.

\textbf{Note:}

Since $\phi$ is not self-framed, then the resulting $\wlp{assert\ \phi_a}{\phi}$ will not necessarily be self-framed. This is okay due to ? accounting for the lack of self-framing and since $\wlp{assert\ \phi_a}{\phi}$ does not lose any information from $\phi_a$ or $\phi$ (in fact, it gains information with respect to $\phi_a$). So, allow this call to $\wlp{assert\ \phi_a}{\fphi}$ even though $\phi$ is not self-framed.

\textbf{Important cases to consider:}

Similar to the ones for the non-gradual version; replacing $\fphi$ with $\phi$.

\vspace{0.5cm}
%
%$\twlp{release \ \phi_a}{? \ast \phi} = $
%
%\vspace{0.5cm}
%
%$\twlp{hold \ \phi_a \ \{s\}}{? \ast \phi} = $
%
%\vspace{0.5cm}

\textbf{Note:} The static parts of gradual formulas do not need to be self-framed unless the gradual formula is completely precise (completely static). The imprecision can account for the framing.

\section{Gradual formula implication and satisfiability}

\subsection{Gradual formula implication implementation advice}

\[ \inferrule*[right=ImplStatic]
   {\fphi_1 \imp static(\tphi_2) }
   {\fphi_1 \timp \tphi_2}
\]

\[
\inferrule*[right=ImplGrad]
   {\fphi \in \satdef \\ \fphi \imp \phi_1 \\ \fphi \imp static(\tphi_2)}
   {? \ast \phi_1 \timp \tphi_2}
\]

$\implsdef$ should be implemented directly for determining gradual formula implication. $\implgdef$ should be implemented by checking $\fphi$ is satisfiable, is self-framed, implies $\phi_1$, and implies $static(\tphi_2)$ and following this advice for determining $\fphi$ (determining $\fphi$ is similar to computing $\wlp{assert\ \phi_a}{\fphi}$, so some of the detailed advice is left out since it is mentioned above):

\begin{enumerate}
\item Start with a list of the accessibility predicates in $\phi_1$ and $static(\tphi_2)$
\item Add accessibility predicates to this list for self-framing $\phi_1$ and $static(\tphi_2)$, as some may be missing since $\phi_1$ and $static(\tphi_2)$ is/may not be self-framed
	\begin{enumerate}
	\item Duplicates (including those due to aliasing) are fine, because they will be removed later
	\item But, not including obvious duplicates will improve performance
	\end{enumerate}
\item Add aliasing information from the actual conjoined accessibility predicates in $\phi_1$ and $static(\tphi_2)$ to the non-accessibility parts of $\phi_1$ and $static(\tphi_2)$ respectively. In other words, the accessibility predicates in $\phi_1$ ($static(\tphi_2)$) separated by the separating conjunction indicate that if the fields ultimately being accessed within two or more of them are the same, then the objects being dereferenced in those field accesses do not alias, so add this information to the non-accessibility part of $\phi_1$ ($static(\tphi_2)$) to not lose information
\item Determine the list of aliases and the list of non-aliases to each variable or field access dereferenced to a final field value in the current list of accessibility predicates using $\mid \phi_1 \mid \ast \mid static(\tphi_2) \mid$
\item Remove duplicate accessibility predicates, including accessibility predicates which are duplicate due to aliasing (the list of aliases and list of non-aliases for each variable and field access dereferenced to a final field value in an acc predicate should help)
	\begin{enumerate}
	\item If it is unknown whether two objects alias due to missing or conflicting information in $\phi_1$ and $static(\tphi_2)$, then assume (for now) that they do not alias
	\end{enumerate}
\item Then $\fphi = \fphi_{acc} \ \ast \mid \phi_1 \mid \ast \mid static(\tphi_2) \mid$, where $\fphi_{acc}$ is the list of non-duplicated accessibility predicates conjoined with the separating conjunction
\end{enumerate}

\subsection{Gradual formula satisfiability implementation advice}
For $\tphi$, if $\tphi = \fphi$, then check satisfiability as usually for fully-precise formulas. If $\tphi = \ ? \ast \phi$, then $? \ast \phi$ is satisfiable iff $\phi$ is satisfiable. Also, there is no need to modify $\phi$ to be self-framed before checking satisfiability. Any attempt to modify $\phi$ to be self-framed would not affect satisfiability, as this modification should ensure the resulting formula is satisfiable if $\phi$ is satisfiable by relying on existing information in $\phi$. If $\phi$ was unsatisfiable, then this modification should not be made, since it relies on information in $\phi$ to be correct. But, if it was made it would not make the resulting formula satisfiable.

% Any additions to $\phi$ from $?$ should not affect satisfiability, as the additions should be made only if the resulting formula is satisfiable if $\phi$ was satisfiable, and if $\phi$ was unsatisfiable, then it would not become satisfiable with the addition of more information from $?$.

In general, it is recommended to check self-framing and satisfiability separately, ie. a formula can be satisfiable and not self-framed.

\end{document}
