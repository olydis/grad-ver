\documentclass {article}

\usepackage[letterpaper, margin=1.0in]{geometry}
\usepackage{amsthm, amsmath, amssymb, stmaryrd}
\usepackage{plstx, listproc}
\usepackage{titling}

\newtheorem{theorem}{Theorem}[section]
\newtheorem{lemma}[theorem]{Lemma}


\title {Verifier Core Language BNF Grammar}
\author {Jenna Wise, Johannes Bader, Henry Blanchette, Jonathan Aldrich, \'{E}ric Tanter}
\date {\today}

%% Commands
\newcommand{\code}{\texttt} % code text

\newcommand{\stararrow}{ - \! \ast \ }
\newcommand{\eif}[3]{\code{if} \ ( #1 ) \ \{ #2 \} \ \code{else} \ \{#3\}}
\newcommand{\tphi}{\widetilde{\phi}}

\begin{document}

\setlength{\droptitle}{-7em}

\maketitle

\begin{figure}[ht!]
\begin{plstx}
  %
  % names
  %
  *(variables)       : x,y,z  [\in] \mathit{VAR} \\
  *(values)          : v      [\in] \mathit{VAL} \\
  *(expressions)     : e      [\in] \mathit{EXPR} \\
  *(statements)      : s      [\in] \mathit{STMT} \\
  *(object Ids)      : o      [\in] \mathit{LOC} \\
  *(field names)     : f      [\in] \mathit{FIELDNAME} \\
  *(method names)    : m      [\in] \mathit{METHODNAME} \\
  *(class names)     : C,D    [\in] \mathit{CLASSNAME} \\
  *(predicate names) : \alpha [\in] \mathit{PREDNAME} \\
  %
  % structures
  %
  : P                 ::= \overline{cls} \ s \\
  : \mathit{cls}      ::= \code{class} \ C \ \code{extends} \ D \ \{ \overline{\mathit{field}} \ \overline{\mathit{pred}} \ \overline{\ \mathit{method}} \} \\
  : \mathit{field}    ::= T \ f; \\
  : \mathit{pred}     ::= \code{predicate} \ \alpha_C(\overline{T \ x}) = \tphi \\
  : T                 ::= \code{int} | \code{bool} | C | \top \\
  % : \mathit{method}   ::= T \ m (\overline{T \ x}) \ \textit{dynamic contract} \ \textit{static contract} \ \{ s \} \\
  : \mathit{method}   ::= T \ m (\overline{T \ x}) \ \code{dynamically} \ \mathit{contract} \ \code{statically} \ \mathit{contract} \ \{ s \} \\
  : \mathit{contract} ::= \code{requires} \ \tphi \ \code{ensures} \ \tphi \\
  : \oplus            ::= + | - | \ast | \backslash | \&\& | \mid\mid \\
  : \odot             ::= \neq | = | < | > | \leq | \geq \\
  : s                 ::= \code{skip}
                         | s_1 \ ; \ s_2
                         | T\ x | x := e
                         | \code{if} \ (e) \ \{s_1\} \ \code{else} \ \{s_2\}
                         | \code{while} \ (e) \ \code{invariant} \ \tphi \ \{ s \}
                         | x.f := y
                         | x := \code{new} \ C
                         | y := z.m(\overline{x})
                         | y := z.m_C(\overline{x})
                         | \code{assert} \ \phi
                         | \code{release} \ \phi
                         | \code{hold} \ \phi \ \{ s \}
                         | \code{fold} \ A
                         | \code{unfold} \ A
                         \\
  : e                 ::= v | x | e \oplus e | e \odot e | e.f \\
  : x                 ::= \code{result} | id | \code{old}(id) | \code{this} \\
  : v                 ::= n | o | \code{null} | \code{true} | \code{false} \\
  : A                 ::= \alpha (\overline{e}) | \alpha_C (\overline{e}) \\
  : \circledast       ::= \land | \ast \\
  : \phi              ::= e
                        | A
                        | \code{acc}(e.f)
                        | \phi \circledast \phi
                        | (\code{if} \ e \ \code{then} \ \phi \ \code{else} \ \phi)
                        | (\code{unfolding} \ A \ \code{in} \ \phi)
                        \\
  : \tphi             ::= \phi | ? \ast \phi \\
\end{plstx}
%\caption{$\svl$ Syntax}
%\label{svl-syntax-fig}
\end{figure}

\end{document}
